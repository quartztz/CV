\documentclass[10pt]{article}

\usepackage[margin=0.75in]{geometry}
\usepackage{changepage}
\usepackage{hyperref}
\usepackage{amsmath}
\usepackage{xcolor}

\def\labelitemi{\ensuremath{\triangleright}}
\renewcommand{\url}[1]{{\texttt{#1}}}
\renewcommand{\line}[2]{{\vspace{4pt} \large \noindent\textsc{#1} \hfill \small{#2}}\vspace{4pt}}


\begin{document}

  \setlength{\parindent}{0em}


  \begin{center}
    \begin{minipage}[C]{0.66\textwidth}
      \begin{center}
        \huge Jacopo Philip Moretti
      \end{center}
      \textit{IN MA1 student at EPFL, with interests in theorem provers, formal verification, and languages, both natural and programming.}

      18.08.03 :: +41 76 730 67 19 :: Chemin des Triaudes, 4b, 1015 Ecublens
      \vspace{-0.75em}
      \begin{center}
        \href{https://people.epfl.ch/jacopo.moretti}{\url{email}} :: \href{https://github.com/quartztz}{\url{github}} :: \href{https://quartztz.github.io}{\url{website}}
      \end{center}
    \end{minipage}
  \end{center}

  \subsection*{Academic}

  \line{MSc in Computer Science :: EPFL}{09.2024 - 2026}

  Master's Degree in Computer Science at the \textit{Ecole Polytechnique Fédérale de Lausanne} (EPFL); projected end in 2026.

  Fields of interest: Theorem Proving, Programming Languages, Natural Language Processing.

  Relevant courses: \textit{Algorithms II (CS450), Machine Learning (CS433), Formal Verification (CS550)}.
  \vspace{0.5em}

  \line{BSc in Computer Science :: EPFL}{09.2021 - 07.2024}

  Bachelor's degree in Computer Science from the \textit{Ecole Polytechnique Fédérale de Lausanne} (EPFL).

  Fields of interest: Functional Programming, Programming Languages.

  Relevant courses: \textit{Algorithms (CS250), Functional Programming (CS207), Computer Language Processing (CS320), Logique Mathématique (MATH381), Software Enterprise (CS311)}.
  \vspace{0.5em}

  \line{French Baccalauréat}{06.2021}

  \textit{Mention Très Bien avec Félicitations du Jury} from the Lycée Chateaubriand de Rome (18.06/20). Obtained in conjunction with the Italian \textit{Esame di Stato} (100/100).

  \subsection*{Professional}

  \line{Chief Product Officer :: actualia}{08.2024 - \textsc{Present}}

  CPO at \textit{actualia}, a startup creating a personalized press review app. Managed design team and led frontend development.

  \textit{Figma, Flutter}

  \line{Student Assistant :: Software Construction (CS214)}{FW 2023}

  Student Assistant for Profs. Odersky, Pit-Claudel, and Kunčak. Answered 100+ student queries on forums, explained concepts in class, and implemented a workflow for recording and editing 20+ course videos.

  \textit{Scala, LaTeX, Teaching}

  \line{Student Assistant :: Students 4 Students}{2022, 2023}

  Volunteer at \textit{Students 4 Students}, developing classes and problem sets for Calculus, Linear Algebra, and Discrete Mathematics, and tutored during exercise sessions.

  \line{Technical Manager :: Fréquence Banane}{05.2022 - 05.2024}

  Committee member tasked with maintaining the technical (audiovisual and server) infrastructure of an 80+ member student radio. Gained knowledge in writing code for large scale and long term use, managing a technical team, and audio engineering.

  \textit{Python, VueJS, Astro}

  \subsection*{Skills}

  \begin{adjustwidth}{}{}
    \begin{itemize}
      \item[] \textbf{Languages}: French (native), Italian (native), English (fluent), German (basic).
      \item[] \textbf{Programming}: Functional (Scala), Systems (Java, C, Rust), Web (\{Java, Type\}Script, WASM,).
    \end{itemize}
  \end{adjustwidth}

  \subsection*{Awards}

  \textit{Winner of the Battle of The Apps, prize awarded by Prof. Candea for the best project in the Software Engineering course (CS311) at EPFL. The project, ActualIA, was eventually expanded into a startup.}

\end{document}
