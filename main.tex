\documentclass[11pt]{article}

\usepackage[margin=1in]{geometry}
\usepackage{changepage}
\usepackage{hyperref}
\usepackage{amsmath}

\def\labelitemi{\ensuremath{\triangleright}}
\renewcommand{\url}[1]{{\texttt{#1}}}
\renewcommand{\line}[2]{{\vspace{4pt} \large \noindent\textsc{#1} \hfill #2}\vspace{4pt}}

\begin{document}
  \begin{center}
    \huge Jacopo Philip Moretti
  \end{center}

  \begin{adjustwidth}{1in}{1in}
    \textit{Étudiant en IN MA1 à l'EPFL, intéréssé par l'informatique théorique, les systèmes de preuve, et les langages, naturels et de programmation.}

    \noindent 18.08.03 :: +41 76 730 67 19 :: Chemin des Triaudes, 4b, 1015 Ecublens
    
    \noindent \href{https://people.epfl.ch/jacopo.moretti}{\url{jacopo.moretti@epfl.ch}}
    
    \noindent \href{https://github.com/quartztz}{\url{github}}
  \end{adjustwidth}

  \section*{Academic}

  \line{MSc en Informatique - EPFL}{2024 - 2026}

  En première année de Master d'Informatique à l'EPFL. Année de complétion prévue: 2026. Champs d'intérêt: Systèmes de preuve, \textit{Large Language Models}.  
  \vspace{1em}

  \line{BSc En Informatique- EPFL}{09.2021 - 07.2024}

  Bachelor d'informatique de l'\textit{Ecole Polytechnique Fédérale de Lausanne} (EPFL). Champs d'intérêt: Langages de Programmation, Systèmes de preuve. 

  Cours principaux:
  \begin{itemize}
    \item \textit{Algorithms} (CS250): Introduction aux structures de données, analyse de la complexité, conception d'algorithmes. \textit{Complété (5/6).}
    \item \textit{Functional Programming} (CS207): Introduction aux notions de la programmation fonctionnelle, récursion, lambda calcul et execution symbolique. Pratique du Scala. \textit{Complété (5.75/6).}
    \item \textit{Computer Language Processing} (CS320): Comprendre le pipeline d'un compilateur, implementer ses étapes lors d'un projet en Scala. \textit{Complété (5.5/6).}
    \item \textit{Logique Mathématique} (MATH381): Théorie des modèles et de la démonstration, théorie des ensembles. Syntaxe et sémantique dans la logique du premier ordre. \textit{Complété.}
    \item \textit{The Software Enterprise - from ideas to products} (CS311) : Compétences de programmation en équipe, à la fois techniques (\texttt{git}) et sociales (\textit{Agile programming}, SCRUM), dans le cadre d'un projet sur un semestre. Scored ``Best Project" award with the ActualIA, an LLM-powered news aggregator. \textit{Complété (4.75/6).}

  \end{itemize}

  \line{French Baccalauréat}{June 2021}

  \textit{Mention Très Bien avec Félicitations du Jury} du Lycée Chateaubriand de Rome (18.06/20). Obtenu en même temps que l'\textit{Esame di Stato} (100/100 avec \textit{Félicitations}).

  \section*{Professional}

  \line{Assistant Étudiant:: Software Construction}{Winter Semester, 23/24}

    Assistant Étudiant sous Prof. Pit-Claudel, Prof. Odersky and Prof. Kunčak. J'ai gagné de l'expérience dans l'assistanat, à la fois en ligne et en classe, ainsi que l'enregistrement et le montage de vidéos pour la \href{https://mediaspace.epfl.ch/channel/CS-214+Software+construction/56193}{\url{chaîne Mediaspace du cours}}.

  \line{Assistant Étudiant :: Students 4 Students}{2022, 2023}
  \line{Assistant Étudiant :: Students 4 Students}{2022, 2023}

  Assistant Étudiant pour l'association étudiante \textit{Students 4 Students}: j'ai aidé à la conception de cours et séries d'éxercices en Analyse, Algèbre Linéaire et Mathématiques Discrètes, ainsi qu'à l'assistanat pendant les séances d'exercice.

  \line{Directeur technique :: Fréquence Banane}{Mai 2022 - Mai 2024}

  Membre du comité chargé de maintenir l'infrastructure informatique et audiovisuelle d'une association étudiante à 80+ membres, j'ai gagné des connaissances vis-à-vis de l'introduction et du développement d'outils techniques sur grande échelle, gestion d'équipe, et de l'audiovisuel pour des manifestations festives et de divulgation scientifique.

  \line{Professeur de soutien}{2021-2023}
  \line{Professeur de soutien}{2021-2023}

  Professeur de soutien particulier en Mathématiques, Physique, Informatique et Littérature Française pour des élèves en âge de lycée.

  \section*{Compétences}
  
  \begin{adjustwidth}{1in}{}
    \begin{itemize}
      \item[\textbf{Mathematics}] Self-taught practice of the Coq interactive theorem prover. Particular interest in the fields of algebra and number theory, as well as model and proof theory. 
      \item[\textbf{Programming}] Object oriented and functional programming, as well as basics of web frameworks. Languages \textit{(in order of proficiency and preferred usage)}: Scala, Rust, Python, C, Java. 
      \item[\textbf{Teaching}] LaTeX/Typst typesetting, course conception (theory, exercices), tutoring.
      \item[\textbf{Languages}] Italian: Native \\ French: Fluent \\ English: Fluent
    \end{itemize}
  \end{adjustwidth}

\end{document}
