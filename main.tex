\documentclass[10pt]{article}

\usepackage[margin=0.75in]{geometry}
\usepackage{changepage}
\usepackage{hyperref}
\usepackage{amsmath}
\usepackage{xcolor}

\def\labelitemi{\ensuremath{\triangleright}}
\renewcommand{\url}[1]{{\texttt{#1}}}
\renewcommand{\line}[2]{{\vspace{4pt} \large \noindent\textsc{#1} \hfill \small{#2}}\vspace{4pt}}


\begin{document}

  \setlength{\parindent}{0em}

  \begin{center}
    \begin{minipage}[C]{0.66\textwidth}
      \begin{center}
        \huge Jacopo Philip Moretti
      \end{center}
      \textit{Étudiant en MA1 à l'EPFL, avec des intérêts dans les prouveurs de théorèmes, la vérification formelle et les langages, tant naturels que de programmation.}

      18.08.03 :: +41 76 730 67 19 :: Chemin des Triaudes, 4b, 1015 Ecublens
      \vspace{-0.75em}
      \begin{center}
        \href{https://people.epfl.ch/jacopo.moretti}{\url{email}} :: \href{https://github.com/quartztz}{\url{github}} :: \href{https://quartztz.github.io}{\url{site web}}
      \end{center}
    \end{minipage}
  \end{center}

  \subsection*{Académique}

  \line{MSc en Informatique :: EPFL}{09.2024 - 2026}

  Master en Informatique à l'\textit{École Polytechnique Fédérale de Lausanne} (EPFL); fin prévue en 2026.

  Intérêts: Prouveurs de théorèmes, Compilateurs, Traitement automatique du langage naturel.

  Cours: \textit{Algorithms II (CS450), Machine Learning (CS433), Formal Verification (CS550)}.
  \vspace{0.5em}

  \line{BSc en Informatique :: EPFL}{09.2021 - 07.2024}

  Bachelor en Informatique à l'\textit{École Polytechnique Fédérale de Lausanne} (EPFL).

  Intérêts: Programmation fonctionnelle, Compilateurs.

  Cours: \textit{Algorithms (CS250), Functional Programming (CS207), Computer Language Processing (CS320), Logique Mathématique (MATH381), Software Enterprise (CS311)}.
  \vspace{0.5em}

  \line{Baccalauréat Français}{06.2021}

  \textit{Mention Très Bien avec Félicitations du Jury} du Lycée Chateaubriand de Rome (18.06/20). Obtenu avec le \textit{Esame di Stato} italien (100/100).

  \subsection*{Professionnel}

  \line{Directeur Produit :: actualia}{08.2024 - \textsc{Présent}}

  Directeur Produit chez \textit{actualia}, une startup créant une application de revue de presse personnalisée. Gestion de l'équipe de design et direction du développement frontend.

  \textit{Figma, Flutter}

  \line{Assistant étudiant :: Software Construction (CS214)}{Automne 2023}

  Assistant étudiant pour les Profs. Odersky, Pit-Claudel, et Kunčak. Réponse aux questions de plus de 100 étudiant·e·s sur les forums et en salle, et mise en place d'un flux de montage des vidéos de cours en 24h.

  \textit{Scala, LaTeX, Enseignement}

  \line{Assistant étudiant :: Students 4 Students}{2022, 2023}

  Bénévole chez \textit{Students 4 Students}, développement de cours et de séries d'exercices d'Analyse, d'Algèbre Linéaire et de Mathématiques Discrètes. Tutorat pendant les séances d'exercices.

  \line{Responsable Technique :: Fréquence Banane}{05.2022 - 05.2024}

  Membre du comité chargé de la maintenance de l'infrastructure technique (audiovisuelle et serveur) d'une radio étudiante de plus de 80 membres et 400+ auditeurs mensuels. Direction de la refonte du site web. Acquisition de connaissances en écriture de code pour une utilisation à grande échelle et à long terme, gestion d'une équipe technique et ingénierie audio.

  \textit{Python, VueJS, Astro}

  \subsection*{Compétences}

  \begin{adjustwidth}{}{}
    \begin{itemize}
      \item[] \textbf{Langues} : Italien (natif), Français (courant), Anglais (courant), Allemand (notions de base).
      \item[] \textbf{Programmation} : Fonctionnelle (Scala), Systèmes (Java, C, Rust), Web (\{Java, Type\}Script, WASM).
    \end{itemize}
  \end{adjustwidth}

  \subsection*{Récompenses}

  \textit{Gagnant de la Battle of The Apps, prix décerné par le Prof. Candea pour le meilleur projet dans le cours de génie logiciel (CS311) à l'EPFL. Le projet, ActualIA, a finalement été développé en une startup.}

\end{document}
