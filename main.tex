\documentclass[11pt]{article}

\usepackage[margin=1in]{geometry}
\usepackage{changepage}
\usepackage{hyperref}
\usepackage{amsmath}

\def\labelitemi{\ensuremath{\triangleright}}
\renewcommand{\url}[1]{{\texttt{#1}}}
\renewcommand{\line}[2]{{\vspace{4pt} \large \noindent\textsc{#1} \hfill #2}\vspace{4pt}}

\begin{document}
  \begin{center}
    \huge Jacopo Philip Moretti
  \end{center}

  \begin{adjustwidth}{1in}{1in}
    \textit{IN MA1 student at EPFL, with interests in theorem provers, formal verification, and languages, both natural and programming.}

    \noindent 18.08.03 :: +41 76 730 67 19 :: Chemin des Triaudes, 4b, 1015 Ecublens
    
    \noindent \href{https://people.epfl.ch/jacopo.moretti}{\url{jacopo.moretti@epfl.ch}}
    
    \noindent \href{https://github.com/quartztz}{\url{github}}
  \end{adjustwidth}

  \section*{Academic}

  \line{MSc in Computer Science - EPFL}{2024 - 2026}

  Pursuing a Master's Degree in Computer Science at the \textit{Ecole Polytechnique Fédérale de Lausanne} (EPFL), with projected end in the year 2026. Fields of interest: Theorem Proving, Large Language Models. 
  \vspace{1em}

  \line{BSc in Computer Science - EPFL}{09.2021 - 07.2024}

  Bachelor's degree in Computer Science from the \textit{Ecole Polytechnique Fédérale de Lausanne} (EPFL). Fields of interest: Programming Language Design, Theorem Proving. 

  Relevant courses include:
  \begin{itemize}
    \item \textit{Algorithms} (CS250): Introduction to data structures, runtime analysis, and algorithm design. \textit{Completed (5/6)}
    \item \textit{Functional Programming} (CS207): Introduction to functional programming concepts, recursion analysis, symbolic and theoretical computer science. \textit{Completed (5.75/6).}
    \item \textit{Computer Language Processing} (CS320): Understanding the compiler pipeline, implementing its main steps in the Scala programming language. \textit{Completed (5.5/6).}
    \item \textit{Logique Mathématique} (MATH381): Basic model and proof theory, set axioms, syntax and semantics in first order logic. \textit{Passed.}
    \item \textit{The Software Enterprise - from ideas to products} (CS311) : Collaborative programming, both technical (\texttt{git}) and social (Agile programming, SCRUM ceremonies). Scored ``Best Project" award with the ActualIA, an LLM-powered news aggregator. \textit{Completed (4.75/6).}
  \end{itemize}

  \line{French Baccalauréat}{June 2021}

  \textit{Mention Très Bien avec Félicitations du Jury} from the Lycée Chateaubriand de Rome (18.06/20). Obtained in conjunction with the Italian Esame di Stato (100/100).
  
  \section*{Professional}

  \line{Student Assistant :: Software Construction}{Winter Semester, 23/24}

  Student assistant under Prof. Pit-Claudel, Prof. Odersky and Prof. Kunčak. Gained experience in assisting students, answering student questions on the forum and explaining concepts on blackboard. Implemented workflow for course recording and video editing.

  \line{Student Assistant :: Students 4 Students}{2022, 2023}

  Student assistant for the association \textit{Students 4 Students}: helped with the conception of classes and problem sets in Calculus, Linear Algebra and Discrete Mathematics, and tutored in exercise sessions.

  \line{Technical Manager :: Fréquence Banane}{May 2022 - May 2024}

  Committee member tasked with maintaining the technical (audiovisual and computer) infrastructure of an 80+ member student radio. Gained knowledge in writing code for large scale and long term use, managing a technical team, and audio engineering.

  \line{Private Tutoring}{2021-2023}

  Private tutoring in Mathematics, Physics, Computer Science and French Language and Literature to teens of high school age and up.

  \section*{Skills}
  
  \begin{adjustwidth}{1in}{}
    \begin{itemize}
      \item[\textbf{Mathematics}] Self-taught practice of the Coq interactive theorem prover. Particular interest in the fields of algebra and number theory, as well as model and proof theory. 
      \item[\textbf{Programming}] Object oriented and functional programming, as well as basics of web frameworks. Languages \textit{(in order of proficiency and preferred usage)}: Scala, Rust, Python, C, Java. 
      \item[\textbf{Teaching}] LaTeX/Typst typesetting, course conception (theory, exercices), tutoring.
      \item[\textbf{Languages}] Italian: Native \\ French: Fluent \\ English: Fluent
    \end{itemize}
  \end{adjustwidth}

\end{document}
